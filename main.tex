\documentclass{article}
\pdfpagewidth=8.5in
\pdfpageheight=11in
\usepackage{ijcai20}

% Use the postscript times font!
\usepackage{times}
\renewcommand*\ttdefault{txtt}
\usepackage{soul}
\usepackage{url}
\usepackage[hidelinks]{hyperref}
\usepackage[utf8]{inputenc}
\usepackage[small]{caption}
\usepackage{graphicx}
\usepackage{amsmath}
\usepackage{booktabs}
\urlstyle{same}
\usepackage{listings}
\usepackage{xcolor}
\usepackage{array}
\usepackage{longtable}
\usepackage{booktabs, multirow} % for borders and merged ranges
\usepackage{soul}% for underlines
\usepackage[table]{xcolor} % for cell colors
\usepackage{changepage,threeparttable} % for wide tables


\lstset{
basicstyle=\ttfamily\small,
keywordstyle=\color{blue},
commentstyle=\color{gray},
stringstyle=\color{orange},
showstringspaces=false,
breaklines=true,
frame=single,
columns=fullflexible,
language=Python,
captionpos=b,
}

\usepackage{todonotes}

\begin{document}

\title{
    \includegraphics[width=8cm, keepaspectratio]{tudelftlogo.png}\\
    \vspace*{2cm}
    \textbf{
         Evaluating Adaptive Activation Functions in Language Models\\
        {\large Does choice of activation function matter in smaller Langaunge Models?}
    }\\
    \vspace*{1cm}
}

\author{
     \textbf{Filip Ignijić}\\
    \hfill \break
    \textbf{Supervisor: Aral de Moor, Responsible Professors: Maliheh Izadi, Arie van Deursen}\\
    \break
%    \affiliations
    {\large 
        \hfill \break
        EEMCS, Delft University of Technology, The Netherlands
    }\\
}

\date{}

\maketitle
\thispagestyle{empty}

\let\clearpagebackup\clearpage
\renewcommand{\clearpage}{ }

\onecolumn

\vspace*{1.5cm}
\begin{center}
    A Thesis Submitted to EEMCS Faculty Delft University of Technology,\\
    In Partial Fulfilment of the Requirements\\
    For the Bachelor of Computer Science and Engineering\\
    \today
\end{center}

\vspace*{2cm}

\noindent
{\small
Name of the student: Filip Ignijić\\
Final project course: CSE3000 Research Project\\
Thesis committee: prof. dr. Thomas Abeel, dr. Maliheh Izadi, prof. dr. Arie van Deursen, Aral de Moor \\
}
\vfill

\begin{center}
    An electronic version of this thesis is available at http://repository.tudelft.nl/.
\end{center}

\twocolumn
\let\clearpage\clearpagebackup  
\clearpage
\setcounter{page}{1}


\begin{abstract} % PRESENT TENSE!!!
    The rapid expansion of large language models (LLMs) driven by the transformer architecture has introduced concerns about the lack of high-quality training data. This study investigates the role of activation functions in smaller-scale language models, specifically those with approximately 10million (M) parameters, to ensure sustained progress in LLM development despite data limitations. Activation functions, crucial for neural network performance, have evolved significantly, but comprehensive comparisons under consistent conditions remain scarce, especially for smaller parameter count models. This research systematically evaluates traditional and novel activation functions, including learnable variants, and introduces the Kolmogorov-Arnold Network (KAN) to language modeling. Using Hugging Face implementations of GPT-Neo and RoBERTa models, this study assesses performance impacts through the BabyLM evaluation pipeline. 
    TODO: Add results and conclusions.
\end{abstract}

\section{Introduction}
Over the last three decades, researchers have proposed approximately 400 different activation functions \cite{Kunc2024}, suggesting a vast landscape of possibilities for neural network optimization. Historically, models such as those based on the transformer architecture, introduced in "Attention is All You Need", predominantly utilized Rectified Linear Unit (ReLU). However, the landscape began shifting when other activation functions started being considered.

A pivotal moment in the evolution of activation functions in language models was marked by the adoption of the Gaussian Error Linear Unit (GELU), as detailed in "GELU Activation Function in Deep Learning: A Comprehensive Mathematical Analysis and Performance". GELU has become the popular choice for language models and it's also the default activation function RoBERTa and GPT-Neo, implemented by Hugging Face which are the ones I will be using as my baseline. This function's popularity underscores its perceived utility over traditional functions like ReLU in specific contexts, particularly in models with parameters on the scale of hundreds of millions.

Despite these advancements, continuous innovation leads to alternatives like GeGLU, noted for its effectiveness in "Glu variants to improve transformer" also used in last year's winner of the BabyLM challenge[source]. Yet, a significant research gap persists—a comprehensive comparison of multiple activation functions under consistent conditions is notably absent. This gap could be explained by findings from "ReLU Strikes Back", which suggests that the impact of activation functions diminishes as the model size increases, evident in models with over a billion parameters. This also explains the initial move away from ReLU, since all the research on alternatives was done on models with the size of approximately 100 million parameters.

Given these insights, this research will explore the impact of various activation functions on smaller-scale language models with around 10 million parameters. The hypothesis posits that at smaller scales, the choice of activation function is crucial, potentially leading to significant performance variations.

Further, this research will delve into an area of adaptive activation functions. It has been shown that adaptive activation functions outperform static ones in text-to-text machine translation [resource], but there seems to be a lack of further research into adaptive function in language models, likely due to an expected tradeoff between additional trainable parameters and impact on performance. Additionally, recent developments in KAN: Kolmogorov-Arnold Networks [resource] suggest a shift towards using activation functions on edges instead of nodes, but due to its recency, it has yet to be tested on a language model. This research will also experiment with this concept and apply it to language modeling to assess its efficacy at smaller scales.

This paper will structure its discussion starting with a review of historical and current activation functions, followed by methodology, experimental setup, results, and conclusions. By addressing these facets, the study aims to illuminate how different activation functions can enhance or compromise the performance of scaled-down language models, ultimately contributing to the optimization of neural network design.
% Over the last three decades, researchers have proposed approximately 400 different activation functions \cite{Kunc2024}, suggesting a vast landscape of possibilities for neural network optimization. Historically, models such as those based on the transformer architecture, introduced in initial transformer paper \cite{Vaswani2017}, predominantly utilized Rectified Linear Unit (ReLU). However, the landscape began shifting when other activation functions started being considered.

% A pivotal moment in the evolution of activation functions in language models was marked by the introducction of the Gaussian Error Linear Unit (GELU)\cite{Lee2023}. GELU has become the popular choice for language models and it's also the default activation function RoBERTa and GPT-Neo, implemented by Hugging Face which are the ones I will be using as my baseline. This function's popularity underscores its perceived utility over traditional functions like ReLU in specific contexts, particularly in models with parameters on the scale of hundreds of millions.

% Yet, as already mentioned, a significant research gap persists in comparison of activation functions, particularly on models smaller than 100m parameters. This gap could be explained by findings from another paper, which suggests that the impact of activation functions diminishes as the model size increases, evident in models with over a billion parameters \cite{Mirzadeh2023}. This also explains the initial move away from ReLU, since all the research on alternatives was done on models with the size of approximately 100 million parameters.

% Given these insights, this research will explore the impact of various activation functions on smaller-scale language models with around 10 million parameters. The hypothesis posits that at smaller scales, the choice of activation function is crucial, potentially leading to significant performance variations.

% Further, this research will delve into an area of adaptive activation functions. It has been shown that adaptive activation functions outperform static ones in text-to-text machine translation \cite{Rajanand}, but there seems to be a lack of further research into adaptive function in language models, likely due to an expected tradeoff between additional trainable parameters and impact on performance. Additionally, recent developments in KAN: Kolmogorov-Arnold Networks \cite{Liu2024} suggest a shift towards using activation functions on edges instead of nodes, but due to its recency, it has yet to be tested on a language model. This research will also experiment with this concept and apply it to language modeling to assess its efficacy at smaller scales.

% This paper will structure its discussion starting with a review of historical and current activation functions, followed by methodology, experimental setup, results, and conclusions. By addressing these facets, the study aims to illuminate how different activation functions can enhance or compromise the performance of scaled-down language models, ultimately contributing to the optimization of neural network design.

\section{Background and related work}
% ** Quick explanation of activation functions ** NOTE: Fix up this paragraph
% - Activation functions are used to introduce non-linearity to neural networks
% - they are on nodes 

% ** Timeline of activations in LLMs **
% - "Attention is all you need" \cite{Vaswani2017} just used the state of the art activation function at the time, ReLU
% - No siginficant improvements till GELU \cite{Lee2023} was introduced which soon became the default activation function in LLMs
% - Development continues new alternatives presented such as GeGLU \cite{Shazeer2020}

% ** The gap **
% - sigmoind -> (models bigger) -> ReLU -> (models bigger) -> GELU -> (models bigger) -> back to relu
% - as mentioned gap of comprehesive comparison of activation functions especially in smaller models
% - the gap can be explained by findings from a recent paper that suggests move back to ReLU as the impact of activation functions diminishes as the model size increases \cite{Mirzadeh2023}
% ADAPTIVE FUNCTIONS GAP
% - another interesting gap is the lac of research of adaptive activation functions in LLMs, only one paper found \cite{Rajanand} which found ... (insert here)
% - very recent development addressing performace of smaller models is KAN: Kolmogorov-Arnold Networks \cite{Liu2024} which suggests a shift towards using activation functions on edges instead of nodes

% ** KAN bacground and gap **
% - KAN is a new type of neural network that uses activation functions on edges instead of nodes
% - It has been shown to outperform traditional neural networks in some tasks \cite{Liu2024}
% - It seems to be mostly suited for scientific applications such as solving partical differential equations \cite{Liu2024}
% - but at the time of my research no tests on LLMs have been done yet
% - the main benefit is optimizing activation on each edge insted of node, using something called splines 
% - the main drawback is the increased number of trainable parameters

% NOTE: Fix Lee2023 citation for GELU
Activation functions are used to introduce non-linearity into neural networks, allowing them to model complex relationships in the data. They are applied to the nodes of the network and are essential for enabling the network to learn and perform a wide range of tasks, beyond what linear models can achieve.

The famous paper "Attention is All You Need" \cite{Vaswani2017} used the state-of-the-art activation function at the time, ReLU. Since then, no significant improvements were made until the introduction of GELU, which quickly became the default activation function in most of the state-of-the-art LLMs \cite{Lee2023}. The popularity of GELU stems from its ability to enhance model performance without introducing an efficiency overhead. Despite these advancements, continuous innovation leads to alternatives like GeGLU, noted for its effectiveness \cite{Shazeer2020}, also used in last year's winner of the BabyLM challenge \cite{Samuel2023}. However, a recently published paper suggests a return to ReLU \cite{Mirzadeh2023}, adding further confusion to the search for the optimal activation function. Fortunately, it also provides some clues that guide further exploration and motivate this research.

The paper suggests that suggests that the impact of activation functions diminishes as the model size increases, evident in models with over a billion parameters \cite{Mirzadeh2023}. This also explains the initial move away from ReLU, since all the research on activation alternatives was done on models with the size of approximately 100 million parameters. Highlighting this finding further motivates the need to investigate activation functions in 10 million parameter models. The impact of activation functions is expected to be more significant in smaller models, and until now, decisions have been made with the trend of increasing model size in mind.

Furthermore, another gap appears in the research on activation functions with trainable parameters (adaptible activaition functions). The possible explanation for this could be the tradeoff between additional trainable parameters and performance. However, this was primarily studied in larger models. The only paper found on adaptive activation functions in LLMs is by Rajanand et al. \cite{Rajanand}, which found that adaptive activation functions outperform static ones in text-to-text machine translation, a task closely related to language models. This suggests that adaptive activation functions could be beneficial for smaller models, but further research is needed to confirm this hypothesis. Given these insights, this research will explore the impact of various activation functions on smaller-scale language models with around 10 million parameters. Hypothesising that at smaller scales, the choice of activation function is crucial and having learnable parameters could be beneficial.

Kolmogorov-Arnold Networks (KAN) represent a recent development in neural network architecture, where activation functions are applied on edges instead of nodes \cite{Liu2024}. This approach has been shown to outperform traditional neural networks in some tasks, particularly in scientific applications such as solving partial differential equations. However, at the time of this literature review, it has yet to be tested on language models. The primary benefit of KAN is the optimization of activation on each edge using splines. A spline is a piecewise-defined polynomial function used in interpolation and approximation to create smooth curves through a set of points [ADD REFERENCE]. With a spline on each edge, each edge can have its own custom activation function, trained separately and uniquely shaped. In contrast, adaptive activation functions have the same shape but different gradients. Although this comes with the drawback of an increased number of trainable parameters. This research will experiment with applying KAN to language modeling to assess its efficacy at smaller scales, filling the gap in current literature.
\clearpage
\vspace*{\fill}
\begin{figure}[hb]
    \centering
    \includegraphics[width=\columnwidth * 2]{figures/kan-network.png}
    \caption{KAN vs MLP [3]}
    \label{fig:your-label}
\end{figure}
\vspace*{\fill}
\clearpage
\section{Approach}

% ** How am i able to do this **
% - transfomers from layers
% - activation functions are used in FFN layer which is implementation of MLP

% ** Motivate choices of activation functions **
% - to compare activation functions I will take existing Hugging face implementation of GPT-NEO and roBERTa and replace the activation function with the one I want to test
% - according to \cite{Mirzadeh2023} I want to test how much worse ReLU is, then compare with relu with learnable parameters PReLU (explain prelu)
% - according to [reference to swish paper] I want to test swish, to get baseline start with SiLU then compare with swish (explain swish)
% - according to [reference to GeGLU paper] I want to test GeGLU, to get baseline start with GELU then compare with GeGLU (explain GeGlU)
% - parametrize GELU with learnable to comapre against GELU (explain how i parametrized gelu)
% - if all that successfull make parametrized GeGLU
% - lastly to compare KAN network against all of the above 

% ** How to kan **
% - implementation exists by the authors of the paper 
% - seems to be running into a bunch of issues and is very slow
% - using efficient kan impelementation isntead
% - have to decide on some paratmees grid: int the number of grid intervals and k: the order of piecewise polynomial, will use 3 and 5 based on the paper 

Transformers are composed of multiple layers, each playing a crucial role in processing input data and generating meaningful representations. Among these layers, the Feed-Forward Neural Network (FFN) layer typically consists of two linear transformations with an activation function in between, effectively forming a simple Multi-Layer Perceptron (MLP). This structure allows the default activation function to be switched out with different activation functions for testing, enabling a direct comparison of their performance while keeping everything else the same. To explore the effectiveness of various activation functions, I will modify existing implementations of prominent models like GPT-NEO and RoBERTa from the Hugging Face library

\subsection{Choosing activation functions}
% The activation functions that will be evaluated are the following: ReLU, SiLU, Swish, PReLU, GELU, GEGLU, learnable GELU and learnable GEGLU. Additionaly the KAN network will be compared against all of those options. 
The activation functions to be evaluated are: ReLU, SiLU, Swish, PReLU, GELU, GEGLU, learnable GELU, and learnable GEGLU. Additionally, the KAN network will be compared against all of these options.

\subsubsection{GELU}
Currently, the most popular activation function in LLMs is also used as the default activation function in the baseline models GPT-NEO and roBERTa. It is a smooth approximation of ReLU, originally defined as \(\text{GELU}(x) = x \cdot \Phi(x)\), where \(\Phi(x)\) is the Cumulative Distribution Function for the Gaussian Distribution. or optimization purposes, since calculating \(\Phi(x)\) is computatinally expensive, it is instead calculated with the tanh approximation as:
\(\text{GELU}(x) = 0.5x \left(1 + \tanh\left(\sqrt{\frac{2}{\pi}} \left(x + 0.044715x^3\right)\right)\right)\) \cite{Hendrycks2023}. This function will be used as a baseline for comparison with all other activations.
[add figure of GELU]

\subsubsection{ReLU and PReLU}
\textit{ReLU} was considered state-of-the-art at the time of the original transformer paper \cite{Vaswani2017}, but has since been surpassed by other activation functions. Baseline models are implemented using the PyTorch library, which prvides an implementation for the ReLU activation function [RELU citation from docs], which will be used in this research. 

\[
\text{ReLU}(x) = \max(0, x)
\]

This implementation will be used as a baseline for comparison with \textit{PReLU} and \textit{GELU}. The comparison with GELU is motivated by the findings of \citet{Mirzadeh2023}, which suggest that the use of ReLU is acceptable as the impact of activation functions diminishes with increasing model size. This research aims to determine how much worse (if at all) ReLU is compared to GELU when used with smaller models.


\textit{PReLU} is a variant of ReLU that includes learnable parameters. This allows the activation function to learn the optimal slope for negative values. The PReLU function is defined as \[\text{PReLU}(x) = \max(0, x) + a \min(0, x)\]
[add figure of ReLU]

\subsubsection{SiLU and Swish}

\subsubsection{GELU and Learnable GELU}

\subsubsection{GEGLU and Learnable geglu}

Motivate choices of activation functions
To explore the effectiveness of various activation functions, I will modify existing implementations of prominent models like GPT-NEO and RoBERTa from the Hugging Face library. According to Mirzadeh et al. (2023), I am interested in evaluating how much worse ReLU performs compared to PReLU, which includes learnable parameters. Additionally, I will test the Swish activation function by starting with SiLU (a baseline version of Swish) and then comparing it to Swish. Furthermore, I will investigate GeGLU by comparing it to the standard GELU activation function. To thoroughly understand the impact, I will also parameterize GELU with learnable components and benchmark it against its standard form. Finally, I will compare these findings against the performance of the KAN network to establish a comprehensive understanding.

The implementation of KAN, as described by the authors of the paper, is available but has shown to be problematic and slow. To address these issues, I will utilize an efficient KAN implementation. This approach requires careful selection of certain parameters, specifically the number of grid intervals (int) and the order of piecewise polynomials (k). Based on recommendations from the paper, I will set these parameters to 3 and 5, respectively. This configuration aims to balance performance and computational efficiency, ensuring that the KAN implementation is both effective and practical for the experiments.
%\input{sections/methodology_or_problem_description}
%
%\input{sections/your_contribution}
%
% Define the research questions clearly. Discuss the specific configurations you have settled on, and
% elaborate on the data sets you have utilized. Document the evaluation setup and define the metrics.
% ● 4.1 Research Questions
% ● 4.2 Dataset
% ● 4.3 Models
% ● 4.4 Evaluation Setting and Metrics.
% Evaluation Metrics: You should mention the components of the BabyLM evaluation pipeline; but can defer
% explaining each (fine-tune) task in detail to the BLiMP/GLUE/SuperGLUE papers. However, you should
% describe the process each of these components uses to return a score (i.e. how is the final score
% computed?)
% Evaluation Settings: hardware.
% ● 4.5 Configuration and Implementation Details

\section{Experimental setup}

\subsection{Research questions}
Through the experiment, we aim to answer the following research questions:
\begin{itemize}
    \item \textit{Is the choice of activation function relevant to the performance of smaller models with 10M parameters?}  We will compare the pre-train baselines models and models using \textit{SiLU} and \textit{ReLU} activation functions and compare the results of evaluation on BabyLM evaluation pipeline \cite{Warstadt2023}.
    \item \textit{How does the addition of learnable parameters to the activation function improve the performance of the model?} We will modify static activation functions to include learnable parameters, pre-train them, and evaluate them on the BabyLM evaluation pipeline \cite{Warstadt2023}.
    \item \textit{Do FFNs using KAN-networks outperform FFNs using MLP networks on performance/computational price metric?} We will use efficient-kan implementation of KAN-networks \cite{efficient-kan}, pre-train them and evaluate them on BabyLM evaluation pipeline \cite{Warstadt2023}.
\end{itemize}

\subsection{Dataset}
We use the TinyStoreis dataset for pre-training. It's a dataset of short stories, that contain words that a typical 4-year-old would likely understand, generated by GPT-3.5 and GPT-4. It has been shown that they can be used to train LMs that are around 10M parameters and can still generate coherent stories. \cite{Eldan2023}.

\subsection{Models}
As already mentioned, we use Hugging face implementations of GPTNeoForCausalLM \cite{huggingfaceNEO} and RobertaForMaskedLM \cite{huggingfaceRoberta} models. GptNEO is GPT2- decoder model, while roBERTa is based on Google's BERT model from 2018. We use these models as baselines for our experiments.
\begin{table*}[h!]
    \centering
    \begin{tabular}{|l|c|c|}
    \hline
    \textbf{Parameter} & \textbf{GPT Neo} & \textbf{RoBERTa} \\ \hline
    \textbf{Embedding Parameters} & & \\ \hline
    Vocab Size & 10,000 & 10,000 \\ \hline
    Hidden Size & 512 & 512 \\ \hline
    Max Position Embeddings & 512 & 513 \\ \hline
    \textbf{Blocks (Attention \& FFN)} & & \\ \hline
    Number of Layers & 2 & 2 \\ \hline
    Attention Types & [[["global", "local"], 1]] & N/A \\ \hline
    Number of Attention Heads & 4 & 4 \\ \hline
    Window Size & 256 & N/A \\ \hline
    Intermediate Size & 1024 & 1024 \\ \hline
    \end{tabular}
    \caption{Comparison of Parameters for GPT Neo and RoBERTa}
\end{table*}
\\

\textit{NOTE: Update the table with changes to parameters made in GeGLU and KAN implementations}

The final parameter counts for the implemented models can be seen in the table below:
\begin{table}[h!]
    \centering
    \begin{tabular}{|l|c|}
    \hline
    \textbf{Model Name} & \textbf{Number of Parameters} \\ \hline
    BERT-GeGlu & 10.0M \\ \hline
    GPT-GeGlu & 10.0M \\ \hline
    BERT-Learnable-GEGLU & 10.0M \\ \hline
    BERT-Learnable-GELU & 9.0M \\ \hline
    GPT-Learnable-GEGLU & 10.0M \\ \hline
    BERT-Learnable-GELU & 9.0M \\ \hline
    GPT-Learnable-GELU & 9.0M \\ \hline
    GPT-ReLU & 9.0M \\ \hline
    BERT-ReLU & 9.0M \\ \hline
    BERT-Learnable-GELU & 9.0M \\ \hline
    GPT-Learnable-GELU & 9.0M \\ \hline
    BERT-Swish & 9.0M \\ \hline
    GPT-Swish & 9.0M \\ \hline
    BERT-PReLU & 9.0M \\ \hline
    GPT-PReLU & 9.0M \\ \hline
    GPT-KAN & 11.0M \\ \hline
    BERT-KAN & 11.0M \\ \hline
    GPT-SiLU & 9.0M \\ \hline
    BERT-SiLU & 9.0M \\ \hline
    GPT-base & 9.0M \\ \hline
    BERT-base & 9.0M \\ \hline
    \end{tabular}
    \caption{Model Names and Number of Parameters}
\end{table}

\subsection{Evaluation Setting and Metrics}
We use the BabyLM evaluation pipeline \cite{Warstadt2023} to evaluate the models. The pipeline consists of three components: BLiMP, GLUE and SuperGLUE. BLiMP is a benchmark for evaluating the linguistic capabilities of language models, consisting of 17, each specific in syntax, morphology or semantics. Models are evaluated zero-shot, by comparing the probabilities of the sequences in a minimal pair, under the assumption that the acceptable sequence will be considered more likely than its unacceptable counterpart. The final score is computed as an average of those 17 metrics. \cite{Warstadt2023blimp} \cite{warstadt-etal-2023-findings}.  GLUE and SuperGLUE are benchmarks for evaluating the performance of language models on a variety of natural language understanding tasks. GLUE consists of 9 tasks, while SuperGLUE consists of 8 tasks. The final score is computed as the average of the scores 7/9 GLUE tasks and 3/8 SuperGLUE tasks \cite{Wang2019} \cite{Wang2020}.

\subsection{Hardware}
All the models were trained and evaluated on a single NVIDIA A100 GPU with 4 CPUs and 24GB of memory on DelftBlue cluster.
% Comparisons
% Gelu (baseline) with Adaptive gelu
% ReLU with PReLU (Adaptive reslu)
% SiLU with Swish
% ReLU with GeLu
\section{Results} % PAST TENSE!!!
\label{sec:results}
\textbf{ToDo: Fix the postition of all those tables bellow.}
All results of evaluations on BLiMP and GLUE datasets are shown in Table \ref{table:all-results}.\\
After performing bootstrap resampling for 10 000 samples, we calculated the mean difference and confidence interval to make each of the comparisons mentioned in section \ref{sec:experimental_setup}.\\
Comparison of Baseline (GELU) model with Adaptive GELU model showed that the mean difference is negative for both BLiMP and GLUE benchmarks, showing that the Adaptive GELU model performed slightly better than the Baseline (GELU) model. However the confidence intervals for both benchmarks contain zero, which means that the difference is not statistically significant. The results are shown in Table \ref{tab:comparison}.\\\\
Boostrap comparing static activations ReLU and SiLU with their adaptive counterparts PReLU and Swish showed that the mean difference is positive for both BLiMP and GLUE benchmarks, showing that the static activation functions performed slightly better than the adaptive activation functions. However the confidence intervals for both benchmarks contain zero, which means that the difference is not statistically significant. The results are shown in Table \ref{tab:comparison-static-adaptive}.\\\\
The highest GLUE score was achieved by the BERT-KAN model with 63.65, the highest BLiMP score was achevied by the GPT-KAN model with 63.43. \textbf{ToDo: Further discuss after all statistical tests are done.}\\\\
Across all models, the training times were relatively consistent, with two outliers: KAN network models and ReLU models. The prolonged training time for the ReLU models might be attributed to the busy DelftBlue nodes, whereas the extended time for the KAN models is expected. We will compute the standard deviations for all models once we obtain all the results. Excluding the two outliers, the standard deviation in training times was 3 minutes and 33 seconds.
\textbf{ToDO: Add results of other comparisons. (evaluations still running on DelftBlue)}

\begin{table}[h]
    \centering
    \begin{tabular}{|l|c|c|}
    \hline
     & \textbf{Blimp} & \textbf{Glue} \\ \hline
    \textbf{Mean difference} & -0.135 & -0.878 \\ \hline
    \textbf{Confidence intervals} & $[-1.16, 0.636]$ & $[-2.674, 0.604]$ \\ \hline
    \end{tabular}
    \caption{Mean differences of combined GPT-Neo and RoBERTa scores for Basline (GELU) and Adaptive (GELU) models}
    \label{tab:comparison}
\end{table}

\begin{table}[h]
    \centering
    \begin{tabular}{|l|c|c|}
    \hline
     & \textbf{Blimp} & \textbf{Glue} \\ \hline
    \textbf{Mean difference} & 0.442 & 1.355 \\ \hline
    \textbf{Confidence intervals} & $[-1.85,  2.3]$ & $[-0.35,  3.05]$ \\ \hline
    \end{tabular}
    \caption{Mean differences of combined GPT-Neo and RoBERTa scores for Static and Adaptive activation functions.}
    \label{tab:comparison-static-adaptive}
\end{table}

\begin{table}[!htp]\centering
    \scriptsize
    \begin{tabular}{lrrrrr}\toprule
    \textbf{Model} &\textbf{Seed} &\textbf{Blimp} &\textbf{Glue} &\textbf{Time} \\\cmidrule{1-5}
    BERT Baseline GELU &42 &54.94 &49.22 & \\
    BERT Baseline GELU &2 &59.5 &59.9 &1h 41m 32s \\
    BERT Baseline GELU &3 &59.6 &56.6 &1h 41m 4s \\
    BERT Baseline GELU &4 &59.3 &57 &1h 41m 5s \\
    BERT Baseline GELU &5 &59.1 &57.5 &1h 42m 21s \\
    GPT Baseline GELU &42 &59.05 &58.22 & \\
    GPT Baseline GELU &1 &58 &58.5 &1h 43m 35s \\
    GPT Baseline GELU &2 &56.6 &57.8 &1h 45m 19s \\
    GPT Baseline GELU &3 &56.6 &59.1 &1h 41m 44s \\
    GPT Baseline GELU &4 &59.2 &60.5 &1h 40m 12s \\
    GPT Baseline GELU &5 &57.4 &59.6 &1h 42m 32s \\
    BERT Adaptive GELU &42 &59.4 &57.1 &1h 58m 39s \\
    BERT Adaptive GELU &1 &59.8 &56.2 &1h 45m 56s \\
    BERT Adaptive GELU &2 &59.3 &59.7 &1h 45m 55s \\
    BERT Adaptive GELU &3 &59 &57 &1h 45m 35s \\
    BERT Adaptive GELU &4 &58.2 &57 &1h 45m 56s \\
    BERT Adaptive GELU &5 &59.8 &59.2 &1h 45m 39s \\
    GPT Adaptive GELU &42 &56.8 &60.2 &1h 43m 41s \\
    GPT Adaptive GELU &1 &56.3 &58.8 &1h 41m 26s \\
    GPT Adaptive GELU &2 &57.4 &58.7 &1h 42m 53s \\
    GPT Adaptive GELU &3 &58.7 &59.6 &1h 41m 11s \\
    GPT Adaptive GELU &4 &56 &59.2 &1h 41m 27s \\
    GPT Adaptive GELU &5 &58.3 &59.3 &1h 41m 19s \\
    \toprule \textbf{Other models (not evaluated on multiple seeds yet) } & & & & \\ \midrule
    BERT ReLU &42 &58.5 &57.2 &2h 24m 46s \\
    BERT PReLU &42 &57 &57.6 &1h 47m 27s \\
    GPT ReLU &42 &56.6 &59.9 &2h 17m 4s \\
    GPT PReLU &42 &59.2 &56.1 &1h 42m 28s \\
    BERT SiLU &42 &58.3 &57.5 &1h 45m 4s \\
    BERT Swish &42 &58 &57.8 &1h 43m 8s \\
    GPT SiLU &42 &56.3 &59.9 &1h 39m 35s \\
    GPT Swish &42 &53.7 &57.6 &1h 39m 22s \\
    BERT KAN &42 &56.69 &63.65 &2h 52m 11s \\
    GPT KAN &42 &63.43 &48.8 &3h 23m 35s \\
    \bottomrule
    \end{tabular}
    \caption{All evaluation results}\label{table:all-results}
\end{table}

\begin{table}[h!]
    \centering
    \begin{tabular}{|l|c|}
    \hline
    \textbf{Model Name} & \textbf{Training Time} \\ \hline
    GPT Baseline GELU & 1h 42m 41s $\pm$ 1m 43s \\ \hline
    BERT Baseline GELU & 1h 41m 27s $\pm$ 32s \\ \hline
    BERT Adaptive GELU & 1h 47m 56s $\pm$ 5m 13s \\ \hline
    GPT Adaptive GELU & 1h 41m 59s $\pm$ 1m 02s \\ \hline
    GPT-ReLU & 2h 17m 4s \\ \hline
    BERT-ReLU & 2h 24m 46s \\ \hline
    BERT-Swish & 1h 43m 8s \\ \hline
    GPT-Swish & 1h 39m 22s \\ \hline
    BERT-PReLU & 1h 47m 27s \\ \hline
    GPT-PReLU & 1h 42m 28s \\ \hline
    KAN2-GPT & 3h 23m 35s \\ \hline
    BERT-kan2 & 2h 52m 11s \\ \hline
    GPT-SiLU & 1h 39m 35s \\ \hline
    BERT-SiLU & 1h 45m 4s \\ \hline
    \end{tabular}
    \caption{Mean training times with standard deviations ToDo: add other SDs and mean times when other evaluations complete.}
    \label{tab:training-times}
\end{table}
\newpage
%
%
%
\section{Discussion} % PRESENT TENSE!!!
TODO: Statistical significance analysis of results and discussion of application of findings.
% bulletpoint list in latex
\begin{itemize}
    \item implications
    \item threats to validity
    \item future work
\end{itemize}
%
\section{Conclusions and Future Work} % PRESENT TENSE!!!
\label{sec:conclusion}
This study aimed to investigate the relevance of activation functions in smaller-scale language models with approximately 10M parameters, addressing three key research questions. The first question explored whether the choice of activation function impacts the performance of these smaller models. The results indicate that the choice of activation function, including GELU and its predecessors, does not significantly affect performance in context of our experiment setup. The second research question examined the benefits of adding learnable parameters to activation functions. Our findings show that parameterizing activation functions, such as with novel adaptive GELU, Swish, and PReLU does not provide a significant performance improvement over their static counterparts. However, due to comupational constraints, the results are inconclusive and further research is needed to explore solidify the results as some of the results contradict well established papers.
 
The third question assessed whether FFNs using Kolmogorov-Arnold Networks (KAN) outperform traditional MLP networks in smaller models. The KAN models consistently underperformed compared to traditional architectures, suggesting that KAN networks may not be a viable alternative for language modeling at this scale or that the implementation used in this study was not optimal.

In summary, while the choice and complexity of activation functions appear to be less critical for smaller language models, it is premature to conclude their irrelevance. Further research is needed to confirm these results and explore other architectural improvements.
%
%\newpage
\section{Appendix}
\label{Appendix}
%If the table is too wide, replace \begin{table}[!htp]...\end{table} with
%\begin{adjustwidth}{-2.5 cm}{-2.5 cm}\centering\begin{threeparttable}[!htb]...\end{threeparttable}\end{adjustwidth}

\begin{table}[!htp]\centering
    \caption{Bootstraped means of GLUE and BLiMP with 95\% confidence intervals}
    \label{tab:bootstraped-means }
    \scriptsize
    \begin{tabular}{lrrrrr}\toprule
    Model &Glue Mean &95\% CI Glue &Blimp Mean &95\% CI Blimp \\\cmidrule{1-5}
    BERT Baseline (GELU) &57.44 &[56.52, 58.73] &58.48 &[56.68, 59.5] \\
    BERT Learnable GELU &57.69 &[56.75, 58.73] &59.25 &[58.78, 59.65] \\
    GPT Baseline (GELU) &58.95 &[58.27, 59.7] &57.82 &[56.97, 58.68] \\
    GPT Learnable GELU &59.30 &[58.92, 59.73] &57.24 &[56.47, 58.03] \\
    BERT ReLU &57.91 &[56.98, 58.87] &58.40 &[57.37, 59.23] \\
    BERT PReLU &58.58 &[57.67, 59.28] &58.37 &[57.70, 59.05] \\
    GPT ReLU &59.41 &[59.0, 59.82] &56.91 &[55.93, 57.8] \\
    GPT PReLU &58.83 &[57.6, 59.82] &57.60 &[56.34, 58.83] \\
    BERT SiLU &57.54 &[57.15, 58.1] &59.16 &[58.75, 59.53] \\
    BERT Swish &57.89 &[57.28, 58,48] &58.26 &[57.71, 58.75] \\
    GPT SiLU &59.05 &[58.4, 59.63] &57.25 &[56.43, 58.2] \\
    GPT Swish &58.36 &[57.95, 58,78] &55.64 &[54.68, 56.64] \\
    GPT KAN &55.15 &[52.54, 56.73] &55.38 &[53.68, 56.47] \\
    \bottomrule
    \end{tabular}
\end{table}
    
    \begin{table}[!htp]\centering
        \caption{Average pre-train times and standard devitations for all the models. \textit{Note:} \textbf{*} \textit{indicates models that were trained on V100s instead of A100s due to time constraints and DelftBlue cluster availability.}}
        \label{tab:average-times}
        \scriptsize
        \begin{tabular}{lrr}\toprule
        Model &Avg. Train. Time \\ \cmidrule{1-2}
        BERT Baseline (GELU) &1h 41m 27s ± 32s \\
        BERT Learnable GELU &1h 47m 56s ± 313s \\
        GPT Baseline (GELU) &1h 42m 41s ± 103s \\
        GPT Learnable GELU &1h 41m 59s ± 102s \\
        *BERT ReLU &2h 25m 23s ± 147s \\
        BERT PReLU &1h 46m 11s ± 46s \\
        *GPT ReLU &2h 20m 27s ± 203s \\
        GPT PReLU &1h 41m 23s ± 40s \\
        BERT SiLU &1h 43m 50s ± 268s \\
        *BERT Swish &2h 20m 4s ± 1086s \\
        GPT SiLU &1h 39m 7s ± 270s \\
        *GPT Swish &2h 31m 4s ± 1032s \\
        *GPT KAN &3h 58m 11s ± 2663s \\
        \bottomrule
        \end{tabular}
        \end{table}


\begin{table}[!htp]\centering
    \caption{Pre-Train times, BLiMP and GLUE scores for all the modes. \textit{Note:} \textbf{*} \textit{indicates models that were trained on V100s instead of A100s due to time constraints and DelftBlue cluster availability.}}
    \label{tab:all-results}
    \scriptsize
    \begin{tabular}{lrrrrr}\toprule
    \textbf{Model} &\textbf{Seed} &\textbf{Blimp} &\textbf{Glue} &\textbf{Time} \\\cmidrule{1-5}
    BERT Baseline (GELU) &42 &54.94 &49.22 &1h 41m 16s \\
    BERT Baseline (GELU) &2 &59.5 &59.9 &1h 41m 32s \\
    BERT Baseline (GELU) &3 &59.6 &56.6 &1h 41m 4s \\
    BERT Baseline (GELU) &4 &59.3 &57 &1h 41m 5s \\
    BERT Baseline (GELU) &5 &59.1 &57.5 &1h 42m 21s \\
    GPT Baseline (GELU) &42 &59.05 &58.22 &1h 42m 44s \\
    GPT Baseline (GELU) &1 &58 &58.5 &1h 43m 35s \\
    GPT Baseline (GELU) &2 &56.6 &57.8 &1h 45m 19s \\
    GPT Baseline (GELU) &3 &56.6 &59.1 &1h 41m 44s \\
    GPT Baseline (GELU) &4 &59.2 &60.5 &1h 40m 12s \\
    GPT Baseline (GELU) &5 &57.4 &59.6 &1h 42m 32s \\
    BERT Learnable GELU &42 &59.4 &57.1 &1h 58m 39s \\
    BERT Learnable GELU &1 &59.8 &56.2 &1h 45m 56s \\
    BERT Learnable GELU &2 &59.3 &59.7 &1h 45m 55s \\
    BERT Learnable GELU &3 &59 &57 &1h 45m 35s \\
    BERT Learnable GELU &4 &58.2 &57 &1h 45m 56s \\
    BERT Learnable GELU &5 &59.8 &59.2 &1h 45m 39s \\
    GPT Learnable GELU &42 &56.8 &60.2 &1h 43m 41s \\
    GPT Learnable GELU &1 &56.3 &58.8 &1h 41m 26s \\
    GPT Learnable GELU &2 &57.4 &58.7 &1h 42m 53s \\
    GPT Learnable GELU &3 &58.7 &59.6 &1h 41m 11s \\
    GPT Learnable GELU &4 &56 &59.2 &1h 41m 27s \\
    GPT Learnable GELU &5 &58.3 &59.3 &1h 41m 19s \\
    *BERT ReLU &42 &58.5 &57.2 &2h 24m 46s \\
    *BERT ReLU &1 &58 &58.7 &2h 22m 40s \\
    *BERT ReLU &2 &56.1 &59.6 &2h 25m 13s \\
    *BERT ReLU &3 &59.2 &56.3 &2h 25m 51s \\
    *BERT ReLU &4 &58.7 &58.8 &2h 29m 52s \\
    *BERT ReLU &5 &59.9 &56.8 &2h 23m 59s \\
    *GPT ReLU &42 &56.6 &59.9 &2h 17m 4s \\
    *GPT ReLU &1 &58.5 &58.7 &2h 16m 42s \\
    *GPT ReLU &2 &56.1 &59.6 &2h 21m 34s \\
    *GPT ReLU &3 &57.4 &59.3 &2h 20m 47s \\
    *GPT ReLU &4 &55 &58.9 &2h 25m 58s \\
    *GPT ReLU &5 &57.9 &60.1 &2h 20m 39s \\
    BERT SiLU &42 &58.3 &57.5 &1h 45m 4s \\
    BERT SiLU &1 &58.9 &58.9 &1h 41m 41s \\
    BERT SiLU &2 &59.8 &57.4 &1h 52m 26s \\
    BERT SiLU &3 &59.6 &57.1 &1h 41m 40s \\
    BERT SiLU &4 &59 &57.3 &1h 40m 43s \\
    BERT SiLU &5 &59.4 &57 &1h 41m 26s \\
    GPT SiLU &42 &56.3 &59.9 &1h 39m 35s \\
    GPT SiLU &1 &59.3 &57.7 &1h 37m 5s \\
    GPT SiLU &2 &55.9 &58.5 &1h 48m \\
    GPT SiLU &3 &57.9 &59.7 &1h 36m 55s \\
    GPT SiLU &4 &57.2 &58.9 &1h 37m \\
    GPT SiLU &5 &56.9 &59.6 &1h 36m 7s \\
    BERT Swish &42 &58 &57.8 &1h 43m 8s \\
    *BERT Swish &1 &57.7 &57.3 &2h 26m 13s \\
    *BERT Swish &2 &57.2 &58.9 &2h 27m 10s \\
    *BERT Swish &3 &58.9 &56.8 &2h 27m 48s \\
    *BERT Swish &4 &59 &58.8 &2h 28m 27s \\
    *BERT Swish &5 &58.8 &57.7 &2h 27m 39s \\
    GPT Swish &42 &53.7 &57.6 &1h 39m 22s \\
    *GPT Swish &1 &55 &58.6 &2h 19m 34s \\
    *GPT Swish &2 &56 &59.2 &2h 20m 51s \\
    *GPT Swish &3 &56.4 &57.9 &2h 21m 43s \\
    *GPT Swish &4 &57.2 &57.8 &2h 21m 30s \\
    *GPT Swish &5 &55.6 &59 &2h 23m 24s \\
    BERT PReLU &42 &57 &57.6 &1h 47m 27s \\
    BERT PReLU &1 &59.8 &59.6 &1h 45m 25s \\
    BERT PReLU &2 &58.2 &58.1 &1h 45m 30s \\
    BERT PReLU &3 &58.7 &56.9 &1h 46m 37s \\
    BERT PReLU &4 &58 &59.5 &1h 45m 54s \\
    BERT PReLU &5 &58.5 &59.8 &1h 46m 14s \\
    GPT PReLU &42 &59.2 &56.1 &1h 42m 28s \\
    GPT PReLU &1 &56.2 &59.1 &1h 41m 7s \\
    GPT PReLU &2 &55.5 &60.5 &1h 40m 40s \\
    GPT PReLU &3 &56.6 &59.5 &1h 41m 50s \\
    GPT PReLU &4 &59.6 &58.2 &1h 40m 48s \\
    GPT PReLU &5 &58.5 &59.6 &1h 41m 24s \\
    GPT KAN &42 &63.43 &48.8 &3h 23m 35s \\
    *GPT KAN &1 &52.7 &56.5 &3h 22m 36s \\
    *GPT KAN &2 &53.8 &56.6 &4h 54m 37s \\
    GPT KAN &3 &54.9 &57 &4h 52m 29s \\
    GPT KAN &4 &54.2 &55.4 &3h 23m 10s \\
    GPT KAN &5 &53.2 &56.7 &3h 52m 41s \\
    \bottomrule
    \end{tabular}
    \end{table}

\end{document}
\section{Reponsible research}
\textbf{TODO:}
Explain how:
\begin{itemize}
    \item Ensured transparency and reproducibility in your research by sharing data, methods, and findings
    openly.
    \item Adhered to principles of scientific integrity, avoiding data manipulation, fabrication, and
    plagiarism
    \item Addressed ethical considerations, disclosed conflicts of interest, and followed guidelines for
    ethical conduct
    \item Incorporated principles and practices from chapters 2 and 3 of the Netherlands Code of Conduct
    for Research Integrity
\end{itemize}

\newpage
\bibliographystyle{plain}
\bibliography{references}
%\newpage
\section{Appendix}
\label{Appendix}
%If the table is too wide, replace \begin{table}[!htp]...\end{table} with
%\begin{adjustwidth}{-2.5 cm}{-2.5 cm}\centering\begin{threeparttable}[!htb]...\end{threeparttable}\end{adjustwidth}

\begin{table}[!htp]\centering
    \caption{Bootstraped means of GLUE and BLiMP with 95\% confidence intervals}
    \label{tab:bootstraped-means }
    \scriptsize
    \begin{tabular}{lrrrrr}\toprule
    Model &Glue Mean &95\% CI Glue &Blimp Mean &95\% CI Blimp \\\cmidrule{1-5}
    BERT Baseline (GELU) &57.44 &[56.52, 58.73] &58.48 &[56.68, 59.5] \\
    BERT Learnable GELU &57.69 &[56.75, 58.73] &59.25 &[58.78, 59.65] \\
    GPT Baseline (GELU) &58.95 &[58.27, 59.7] &57.82 &[56.97, 58.68] \\
    GPT Learnable GELU &59.30 &[58.92, 59.73] &57.24 &[56.47, 58.03] \\
    BERT ReLU &57.91 &[56.98, 58.87] &58.40 &[57.37, 59.23] \\
    BERT PReLU &58.58 &[57.67, 59.28] &58.37 &[57.70, 59.05] \\
    GPT ReLU &59.41 &[59.0, 59.82] &56.91 &[55.93, 57.8] \\
    GPT PReLU &58.83 &[57.6, 59.82] &57.60 &[56.34, 58.83] \\
    BERT SiLU &57.54 &[57.15, 58.1] &59.16 &[58.75, 59.53] \\
    BERT Swish &57.89 &[57.28, 58,48] &58.26 &[57.71, 58.75] \\
    GPT SiLU &59.05 &[58.4, 59.63] &57.25 &[56.43, 58.2] \\
    GPT Swish &58.36 &[57.95, 58,78] &55.64 &[54.68, 56.64] \\
    GPT KAN &55.15 &[52.54, 56.73] &55.38 &[53.68, 56.47] \\
    \bottomrule
    \end{tabular}
\end{table}
    
    \begin{table}[!htp]\centering
        \caption{Average pre-train times and standard devitations for all the models. \textit{Note:} \textbf{*} \textit{indicates models that were trained on V100s instead of A100s due to time constraints and DelftBlue cluster availability.}}
        \label{tab:average-times}
        \scriptsize
        \begin{tabular}{lrr}\toprule
        Model &Avg. Train. Time \\ \cmidrule{1-2}
        BERT Baseline (GELU) &1h 41m 27s ± 32s \\
        BERT Learnable GELU &1h 47m 56s ± 313s \\
        GPT Baseline (GELU) &1h 42m 41s ± 103s \\
        GPT Learnable GELU &1h 41m 59s ± 102s \\
        *BERT ReLU &2h 25m 23s ± 147s \\
        BERT PReLU &1h 46m 11s ± 46s \\
        *GPT ReLU &2h 20m 27s ± 203s \\
        GPT PReLU &1h 41m 23s ± 40s \\
        BERT SiLU &1h 43m 50s ± 268s \\
        *BERT Swish &2h 20m 4s ± 1086s \\
        GPT SiLU &1h 39m 7s ± 270s \\
        *GPT Swish &2h 31m 4s ± 1032s \\
        *GPT KAN &3h 58m 11s ± 2663s \\
        \bottomrule
        \end{tabular}
        \end{table}


\begin{table}[!htp]\centering
    \caption{Pre-Train times, BLiMP and GLUE scores for all the modes. \textit{Note:} \textbf{*} \textit{indicates models that were trained on V100s instead of A100s due to time constraints and DelftBlue cluster availability.}}
    \label{tab:all-results}
    \scriptsize
    \begin{tabular}{lrrrrr}\toprule
    \textbf{Model} &\textbf{Seed} &\textbf{Blimp} &\textbf{Glue} &\textbf{Time} \\\cmidrule{1-5}
    BERT Baseline (GELU) &42 &54.94 &49.22 &1h 41m 16s \\
    BERT Baseline (GELU) &2 &59.5 &59.9 &1h 41m 32s \\
    BERT Baseline (GELU) &3 &59.6 &56.6 &1h 41m 4s \\
    BERT Baseline (GELU) &4 &59.3 &57 &1h 41m 5s \\
    BERT Baseline (GELU) &5 &59.1 &57.5 &1h 42m 21s \\
    GPT Baseline (GELU) &42 &59.05 &58.22 &1h 42m 44s \\
    GPT Baseline (GELU) &1 &58 &58.5 &1h 43m 35s \\
    GPT Baseline (GELU) &2 &56.6 &57.8 &1h 45m 19s \\
    GPT Baseline (GELU) &3 &56.6 &59.1 &1h 41m 44s \\
    GPT Baseline (GELU) &4 &59.2 &60.5 &1h 40m 12s \\
    GPT Baseline (GELU) &5 &57.4 &59.6 &1h 42m 32s \\
    BERT Learnable GELU &42 &59.4 &57.1 &1h 58m 39s \\
    BERT Learnable GELU &1 &59.8 &56.2 &1h 45m 56s \\
    BERT Learnable GELU &2 &59.3 &59.7 &1h 45m 55s \\
    BERT Learnable GELU &3 &59 &57 &1h 45m 35s \\
    BERT Learnable GELU &4 &58.2 &57 &1h 45m 56s \\
    BERT Learnable GELU &5 &59.8 &59.2 &1h 45m 39s \\
    GPT Learnable GELU &42 &56.8 &60.2 &1h 43m 41s \\
    GPT Learnable GELU &1 &56.3 &58.8 &1h 41m 26s \\
    GPT Learnable GELU &2 &57.4 &58.7 &1h 42m 53s \\
    GPT Learnable GELU &3 &58.7 &59.6 &1h 41m 11s \\
    GPT Learnable GELU &4 &56 &59.2 &1h 41m 27s \\
    GPT Learnable GELU &5 &58.3 &59.3 &1h 41m 19s \\
    *BERT ReLU &42 &58.5 &57.2 &2h 24m 46s \\
    *BERT ReLU &1 &58 &58.7 &2h 22m 40s \\
    *BERT ReLU &2 &56.1 &59.6 &2h 25m 13s \\
    *BERT ReLU &3 &59.2 &56.3 &2h 25m 51s \\
    *BERT ReLU &4 &58.7 &58.8 &2h 29m 52s \\
    *BERT ReLU &5 &59.9 &56.8 &2h 23m 59s \\
    *GPT ReLU &42 &56.6 &59.9 &2h 17m 4s \\
    *GPT ReLU &1 &58.5 &58.7 &2h 16m 42s \\
    *GPT ReLU &2 &56.1 &59.6 &2h 21m 34s \\
    *GPT ReLU &3 &57.4 &59.3 &2h 20m 47s \\
    *GPT ReLU &4 &55 &58.9 &2h 25m 58s \\
    *GPT ReLU &5 &57.9 &60.1 &2h 20m 39s \\
    BERT SiLU &42 &58.3 &57.5 &1h 45m 4s \\
    BERT SiLU &1 &58.9 &58.9 &1h 41m 41s \\
    BERT SiLU &2 &59.8 &57.4 &1h 52m 26s \\
    BERT SiLU &3 &59.6 &57.1 &1h 41m 40s \\
    BERT SiLU &4 &59 &57.3 &1h 40m 43s \\
    BERT SiLU &5 &59.4 &57 &1h 41m 26s \\
    GPT SiLU &42 &56.3 &59.9 &1h 39m 35s \\
    GPT SiLU &1 &59.3 &57.7 &1h 37m 5s \\
    GPT SiLU &2 &55.9 &58.5 &1h 48m \\
    GPT SiLU &3 &57.9 &59.7 &1h 36m 55s \\
    GPT SiLU &4 &57.2 &58.9 &1h 37m \\
    GPT SiLU &5 &56.9 &59.6 &1h 36m 7s \\
    BERT Swish &42 &58 &57.8 &1h 43m 8s \\
    *BERT Swish &1 &57.7 &57.3 &2h 26m 13s \\
    *BERT Swish &2 &57.2 &58.9 &2h 27m 10s \\
    *BERT Swish &3 &58.9 &56.8 &2h 27m 48s \\
    *BERT Swish &4 &59 &58.8 &2h 28m 27s \\
    *BERT Swish &5 &58.8 &57.7 &2h 27m 39s \\
    GPT Swish &42 &53.7 &57.6 &1h 39m 22s \\
    *GPT Swish &1 &55 &58.6 &2h 19m 34s \\
    *GPT Swish &2 &56 &59.2 &2h 20m 51s \\
    *GPT Swish &3 &56.4 &57.9 &2h 21m 43s \\
    *GPT Swish &4 &57.2 &57.8 &2h 21m 30s \\
    *GPT Swish &5 &55.6 &59 &2h 23m 24s \\
    BERT PReLU &42 &57 &57.6 &1h 47m 27s \\
    BERT PReLU &1 &59.8 &59.6 &1h 45m 25s \\
    BERT PReLU &2 &58.2 &58.1 &1h 45m 30s \\
    BERT PReLU &3 &58.7 &56.9 &1h 46m 37s \\
    BERT PReLU &4 &58 &59.5 &1h 45m 54s \\
    BERT PReLU &5 &58.5 &59.8 &1h 46m 14s \\
    GPT PReLU &42 &59.2 &56.1 &1h 42m 28s \\
    GPT PReLU &1 &56.2 &59.1 &1h 41m 7s \\
    GPT PReLU &2 &55.5 &60.5 &1h 40m 40s \\
    GPT PReLU &3 &56.6 &59.5 &1h 41m 50s \\
    GPT PReLU &4 &59.6 &58.2 &1h 40m 48s \\
    GPT PReLU &5 &58.5 &59.6 &1h 41m 24s \\
    GPT KAN &42 &63.43 &48.8 &3h 23m 35s \\
    *GPT KAN &1 &52.7 &56.5 &3h 22m 36s \\
    *GPT KAN &2 &53.8 &56.6 &4h 54m 37s \\
    GPT KAN &3 &54.9 &57 &4h 52m 29s \\
    GPT KAN &4 &54.2 &55.4 &3h 23m 10s \\
    GPT KAN &5 &53.2 &56.7 &3h 52m 41s \\
    \bottomrule
    \end{tabular}
    \end{table}

\end{document}

%A rule of thumb for dealing with the literature is the following: scan about 10--20 contributions: read title, abstract, part of introduction and conclusions; categorize contribution; some of these are studied in more depth: completely read about 5 conference papers or equivalent (summarize contribution in own words); of which studied in-depth about 2 conference papers (the student is able to explain in detail and criticize contributions). This may result in 5--20 references, possibly even more if the project is a literature study.


\end{document}

